\textbf{Tugas 2}
\begin{enumerate}
    \item Hamiltonian: $H(x,p)=\dfrac{p^2}{2m}+\dfrac{1}{2}kx^2=E$
    \begin{equation*}
        \begin{split}
            \rightarrow\;H(0,p)&=\dfrac{p^2}{2m}=E\\
            p&=\sqrt{2mE}\\
            H(L,p)&=\dfrac{p^2}{2m}+\dfrac{1}{2}kL^2=E\\
            p&=\sqrt{\dfrac{2mE}{p^2+mkL^2}}
        \end{split}
    \end{equation*}
    dengan 1 dimensi $\rightarrow$ 1N
    \begin{equation*}
        \begin{split}
            \Delta\omega&=\int_{E\ll H\ll E\delta E} dxdx=\int_{E\ll H\ll E\delta E} d\omega\\
            \omega&=\int_{E\ll H\ll E\delta E}L\,dp\\
            &=\int_{\sqrt{2mE}}^{\sqrt{\sfrac{2mE}{p^2+mkL^2}}}Ldp\\
            &=L\left(\sqrt{2mE}\right)\left(\dfrac{1-\sqrt{p^2+mkL^2}}{\sqrt{p^2+mkL^2}}\right)
        \end{split}
    \end{equation*}
    \item 
    \begin{enumerate}[(a)]
        \item dengan probabilitas ensembel
        \begin{equation*}
            P(x)dx=\sum\dfrac{W(\varphi) d\varphi}{\sfrac{dx}{d\varphi}}
        \end{equation*}
        dengan persamaan gemolmbang osilator harmonik
        \begin{equation*}
            \dfrac{dx}{d\varphi}=A\sin(\omega t+\varphi)
        \end{equation*}
        maka,
        \begin{equation*}
            \omega=(\varphi)d\varphi=\dfrac{d\varphi}{2\pi}
        \end{equation*}
        sehingga didapatkan,
        \begin{equation*}
            \begin{split}
                P(x)dx&=\sum\dfrac{W(\varphi) d\varphi}{\sfrac{dx}{d\varphi}}\\
                &=\dfrac{2dx}{2\pi A\sin(\omega t+\varphi)}\\
                &=\dfrac{1}{\pi\sqrt{A^2-x^2}}dx
            \end{split}
        \end{equation*}
        \item Dengan
        \begin{equation*}
            \begin{split}
                x&=A\cos(\omega t+\varphi)\\
                \dfrac{dx}{d\varphi}&=A\sin(\omega t+\varphi)
            \end{split}
        \end{equation*}
        dan dengan
        \begin{equation*}
            \omega(\varphi)\,d\varphi=(2\pi)^{-1}\,d\varphi
        \end{equation*}
        Sehingga didapatkan
        \begin{equation*}
            \begin{split}
                p(x)dx&=\dfrac{2\,dx}{2\pi A\sin(\omega t+\varphi)}\\
                &=\dfrac{1}{\pi\sqrt{A^2-x^2}}\,dx
            \end{split}
        \end{equation*}
        \item Energi sebagai fungsi dari amplitudo
        \begin{equation*}
            \begin{split}
                E&=\dfrac{p^2}{2m}+\dfrac{kx^2}{2}\\
                &=\dfrac{kA^2}{2}
            \end{split}
        \end{equation*}
        Dengan kontur energi pada fase ruang sebanding dengan elips.\\
        Dengan transformasi
        \begin{equation*}
            p\prime^2=\dfrac{p^2}{mk}
        \end{equation*}
        maka didapat lingkaran sebagai kontur energi, sehingga:
        \begin{equation*}
            \rightarrow A^2=x^2+p\prime^2 
        \end{equation*}
        dengan fase ruang volume berada diandata $E$ dan $E+\delta E$ yang direpresentasikan dengan luas $a$ diantara $A$ dan $A+\delta A$ dimana $\delta A$ merupakan fungsi
        \begin{equation*}
            \omega(A)\delta A=2\pi A\delta A
        \end{equation*}
        Untuk menemukan dimana bagian dari ruang fase diantara $x$ dan $dx$, gunakan koordinat polar
        \begin{equation*}
            \begin{split}
                \cos\theta&=\dfrac{x}{A}\\
                d\theta&=\dfrac{dx}{A\sin\theta}=\dfrac{dx}{\sqrt{A^2-x^2}}
            \end{split}
        \end{equation*}
        Sehingga area diantara dua bagian antara $x$ dan $dx$ adalah
        \begin{equation*}
            \begin{split}
                \omega(x,A)\,dx\delta A&=2A\,d\theta\delta A\\
                &=\dfrac{2A\,dx\delta A}{\sqrt{A^2-x^2}}
            \end{split}
        \end{equation*}
        dan probabilitasnya dalah
        \begin{equation*}
            \begin{split}
                P(x)dx&=\dfrac{\omega(x,A)dx\delta A}{\omega(A)\delta A}\\
                &=\dfrac{dx}{\pi\sqrt{A^2-x^2}}
            \end{split}
        \end{equation*}
    \end{enumerate}
    \item 
    \begin{enumerate}[(a)]
        \item 
        \begin{equation*}
            \begin{split}
                F&=F(x,y)\\
                dF&=\dfrac{\partial F}{\partial x}\,dy+\dfrac{\partial F}{\partial y}\,dy\\
                \leadsto&\,\,dF=A\,dx+B\,dy
            \end{split}
        \end{equation*}
        dengan
        \begin{equation*}
            A=\dfrac{\partial F}{\partial x}
        \end{equation*}
        maka, 
        \begin{equation*}
            \dfrac{\partial A}{\partial y}=\dfrac{\partial^2 F}{\partial x\partial y}
        \end{equation*}
        dan 
        \begin{equation*}
            B=\dfrac{\partial F}{\partial y}
        \end{equation*}
        maka,
        \begin{equation*}
            \dfrac{\partial B}{\partial x}=\dfrac{\partial^2 F}{\partial x\partial y}
        \end{equation*}
        sehingga 
        \begin{equation*}
            \dfrac{\partial A}{\partial y}=\dfrac{\partial B}{\partial x}
        \end{equation*}
        \item 
        \begin{equation*}
            \begin{split}
                \int_A^B \,dF&=F(B)-F(A)\\
                &=F(x_N,y_N)- F(x_0,y_0)
            \end{split}
        \end{equation*}
        Lintasan tertutup, maka $X_N\rightarrow X_0$ dan $y_N\rightarrow y_0$
        \begin{equation*}
            \begin{split}
                \oint\,df&=\int_A^B\,dF\\
                &=F(x_0,y_0)- F(x_0,y_0)\\
                &=0
            \end{split}
        \end{equation*}
    \end{enumerate}
\end{enumerate}