\textbf{Tugas 3}
\begin{enumerate}
    \item  
    \begin{enumerate}[(a)]
        \item Jumlah rata-rata molekul untuk setiap kotak\\
        Pada kotak besar
        \begin{itemize}
            \item jumlah Ne $=1000\times\dfrac{3}{4}=750$ molekul
            \item jumlah He $=100\times\dfrac{3}{4}=75$ molekul
        \end{itemize}
        Pada kotak kecil
        \begin{itemize}
            \item jumlah Ne $=1000\times\dfrac{1}{4}=250$ molekul
            \item jumlah He $=100\times\dfrac{1}{4}=25$ molekul
        \end{itemize}
        \item Probabilitas dari 1000 molekul Ne dan 100 molekul N
        \begin{itemize}
            \item molekul Ne pada kotak besar:
            \begin{equation*}
                P_{Ne}=\left(\dfrac{3}{4}\right)^{1000}
            \end{equation*}
            \item molekul Ne pada kotak besar:
            \begin{equation*}
                P_{Ne}=\left(\dfrac{1}{4}\right)^{100}
            \end{equation*}
        \end{itemize}
    \end{enumerate}
    \item 
    \begin{enumerate}[(a)]
        \item Nilai mikrostate
        \begin{equation*}
            \Omega(N,n_{\uparrow})=\dfrac{N!}{n_{\uparrow}!(N-n_{\uparrow})}
        \end{equation*}
        dengan energi medan magnet
        \begin{equation*}
            \begin{split}
                E&=(N-2n_{\uparrow})\mu B\;\;\;\leadsto H=\dfrac{B}{\mu}\\
                &=(N-2n_{\uparrow})\mu^2 B
            \end{split}
        \end{equation*}
        dengan entropi
        \begin{equation*}
            S=k_B\ln\Omega=k_B\ln\left[\dfrac{N!}{n_{\uparrow}!(N-n_{\uparrow})}\right]\leadsto n_{\uparrow}=\dfrac{1}{2}\left(N-\dfrac{E}{\mu^2 H}\right)
        \end{equation*}
        dari formula stirling
        \begin{equation*}
            S=k_B\left[N\ln N-n_{\uparrow}\ln n_{\uparrow}-(N-n_{\uparrow})\ln(N-n_{\uparrow})\right]
        \end{equation*}
        dan dari relasi termodinamik
         \begin{equation*}
             \begin{split}
                 dE&=TdS-mdB\\
                 dS&=\dfrac{1}{T}dE+\dfrac{m}{T}dB
             \end{split}
         \end{equation*}
         Sehingga didapatkan:
         \begin{equation*}
            \begin{split}
                \left(\dfrac{\partial S}{\partial E}\right)&=\dfrac{1}{T}\\
                k_B\dfrac{\partial \ln\Omega}{\partial E}&=\dfrac{1}{T}\;\;\;\leadsto\beta=\dfrac{\partial \ln\Omega}{\partial E}\\
                k_B\beta&=\dfrac{1}{T}
            \end{split}
         \end{equation*}
         maka,
         \begin{equation*}
             \begin{split}
                \dfrac{1}{T}&=k_B\left[-\ln n_{\uparrow}-1+\ln(N-n_{\uparrow}+1)\right]\left(-\dfrac{1}{2\mu^2H}\right)\\
                &=k_B\left[-\ln \left(\dfrac{1}{2}\left(N-
                \dfrac{E}{\mu^2H}\right)\right)+\ln\left(N-\dfrac{1}{2}\left(N-
                \dfrac{E}{\mu^2H}\right)\right)\right]\left(-\dfrac{1}{2\mu^2H}\right)
             \end{split}
         \end{equation*}
         \item 
         \begin{equation*}
            dS=\dfrac{1}{T}dE+\dfrac{M}{T} 
         \end{equation*}
         kemudian didapatkan
        \begin{equation*}
            \left(\dfrac{\partial S}{\partial B}\right)=\dfrac{M}{T}
        \end{equation*}
        maka,
        \begin{equation*}
            \begin{split}
                \dfrac{M}{T}&=k_B\left[-\ln \left(\dfrac{1}{2}\left(N-
                \dfrac{E}{\mu^2H}\right)\right)+\ln\left(N-\dfrac{1}{2}\left(N-
                \dfrac{E}{\mu^2H}\right)\right)\right]\left(\dfrac{E}{2\mu^3H^2}\right)\\
                M(H,T)&=k_BT\left[-\ln \left(\dfrac{1}{2}\left(N-
                \dfrac{E}{\mu^2H}\right)\right)+\ln\left(N-\dfrac{1}{2}\left(N-
                \dfrac{E}{\mu^2H}\right)\right)\right]\left(\dfrac{E}{2\mu^3H^2}\right)
            \end{split}
        \end{equation*}
    \end{enumerate}
    \item Perubahan entropi diberikan oleh:
    \begin{equation*}
        \Delta S=\int dS=\int \dfrac{dQ}{T}
    \end{equation*}
    Dengan sistem $A$ berada pada temperatur lebih rendah dari sistem $A\prime$, sehingga 
    \begin{equation*}
        Q>0\;\;\;\text{dan}\;\;\;T\leq T\prime
    \end{equation*}
    maka didapat:
    \begin{equation*}
        \begin{split}
            \Delta S&=\int \dfrac{dQ}{T}>\int\dfrac{dQ}{T\prime}\\
            &=\dfrac{Q}{T\prime}
        \end{split}
    \end{equation*}
    karena $T\prime$ konstan untuk reservoir panas, maka didapat:
    \begin{equation*}
        \Delta S\geq \dfrac{Q}{T\prime}
    \end{equation*}
\end{enumerate}